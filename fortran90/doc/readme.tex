% Created 2018-03-07 Wed 13:42
% Intended LaTeX compiler: pdflatex
\documentclass[11pt]{article}
\usepackage[utf8]{inputenc}
\usepackage[T1]{fontenc}
\usepackage{graphicx}
\usepackage{grffile}
\usepackage{longtable}
\usepackage{wrapfig}
\usepackage{rotating}
\usepackage[normalem]{ulem}
\usepackage{amsmath}
\usepackage{textcomp}
\usepackage{amssymb}
\usepackage{capt-of}
\usepackage{hyperref}
\usepackage{minted}
\author{Tae Eun Kim}
\date{\today}
\title{2-D vortex patch fortran 90 program}
\hypersetup{
 pdfauthor={Tae Eun Kim},
 pdftitle={2-D vortex patch fortran 90 program},
 pdfkeywords={},
 pdfsubject={},
 pdfcreator={Emacs 25.3.1 (Org mode 9.1.6)},
 pdflang={English}}
\begin{document}

\maketitle
\tableofcontents

\section{Todo's}
\label{sec:org5b38cc3}
\begin{itemize}
\item[{$\boxtimes$}] Useful general modules
\begin{itemize}
\item[{$\boxtimes$}] constants: pi, i, e, line(---\ldots{}), \ldots{}
\item[{$\boxtimes$}] function(vnorm) 1-norm / 2-norm / \sout{infinity-norm}
\item[{$\boxtimes$}] subroutines: linspace (a+(b-a)*i/n, i=0,n) and linspace1 (a+(b-a)*i/n, i=0,n-1)
\item[{$\boxtimes$}] subroutines: timestamp
\end{itemize}
\item[{$\boxtimes$}] subroutines and functions
\begin{itemize}
\item[{$\boxtimes$}] reading and writing data
\item[{$\boxtimes$}] point evaluation
\item[{$\boxtimes$}] velocity (integration) - new version with efficient alternating point quadrature
\item[{$\boxtimes$}] residual (as a function)
\item[{$\boxtimes$}] vsolver -  newton continuation
\item[{$\boxtimes$}] c\(_{\text{0}}\) calculation
\item[{$\boxtimes$}] transfer matrix M calculation - efficient version using fft
\item[{$\boxtimes$}] j(nu) calculation
\end{itemize}
\item[{$\boxtimes$}] newton continuation method
\begin{itemize}
\item[{$\boxtimes$}] LINPACK
\item[{$\boxtimes$}] LAPACK
\item[{$\boxtimes$}] save solution
\end{itemize}
\item[{$\boxminus$}] fft
\begin{itemize}
\item[{$\boxtimes$}] Using fftw
\item[{$\boxtimes$}] double Fourier series: two 1-D FFT \& one 2-D FFT
\item[{$\square$}] FFT based on shifted ponits
\item[{$\square$}] Self-contained code? Is it possible?
\end{itemize}
\item[{$\boxtimes$}] Is it possible to compile files in subdirectory and to save
binary files in a subdirectory? - see below
\item[{$\square$}] Find out why coefficients have discrepancy (\textasciitilde{}1e-12) between matlab and fortran
\item[{$\square$}] plotting using gnuplot
\end{itemize}
\section{Directory organization}
\label{sec:org6d95a01}
\subsection{Project directory}
\label{sec:org46af9f0}
\begin{minted}[]{shell}
tree -d ../
\end{minted}

\subsection{{\bfseries\sffamily ☛ TODO} More specialized and cleaner structure}
\label{sec:org0e31fc4}
See if I can clean it up more using functionalities of \texttt{make}.
\begin{verbatim}
../
├── data
├── doc
├── src
├── bin
├── obj
└── fig

6 directories
\end{verbatim}
\section{Program organization}
\label{sec:org028dfb7}
\subsection{Modules}
\label{sec:org04d2a7d}
\textbf{ISSUE.} When tangled, emacs automatically wraps the module file with \texttt{program main} and \texttt{end program main}, which need to be deleted manually.
\subsubsection{Constants}
\label{sec:org5e7249f}
\begin{itemize}
\item set real kind (single or double precition)
\item pi, i, eps
\end{itemize}
\subsubsection{Basic routines - printing, linspace, etc}
\label{sec:orga952248}
\begin{itemize}
\item\relax [SR] writing complex data
\item\relax [FN] linspace, linspace1, linspaceh
\item\relax [FN] identity matrix, zero array constructor
\item\relax [FN] vector infinity-norm calculator
\item\relax [FN] generating data filename
\item\relax [SR] time stamp
\item\relax [FN] matrix infinity-norm calculator
\item\relax [SR] generating data filename
\end{itemize}
\subsubsection{Notes:}
\label{sec:orgad8991a}
\begin{itemize}
\item When a function defined externally outside the main program, it appears that gfortran compiler wants to have an \texttt{interface} block. In case a function defined in a module is called, it works fine without \texttt{interface} block.
\end{itemize}
\subsection{Main program}
\label{sec:org715b463}
\subsubsection{solver program}
\label{sec:org6e7e368}
\subsubsection{transfer matrix calculation}
\label{sec:org664a940}
\subsubsection{fft program}
\label{sec:orga40ca8c}
\subsubsection{calculation of bounds}
\label{sec:orgf1a23b1}
\subsubsection{testing program}
\label{sec:orgcc15a15}
\subsection{Subroutines and functions}
\label{sec:orgf899ab8}
\subsubsection{Reading data}
\label{sec:org1f5f3da}
\subsubsection{Writing data}
\label{sec:orgaa61298}
\subsubsection{Point evaluation}
\label{sec:org59c08b9}
\subsubsection{Velocity calculation}
\label{sec:org4d5ae3b}
\subsubsection{Residual calculation}
\label{sec:org17ab98d}
\subsubsection{Vortex patch solver (Newton continuation)}
\label{sec:org5ec0355}
\subsubsection{FFT and \$T\$-matrix calculation}
\label{sec:org15010ea}
\section{Data file structure}
\label{sec:org18b491a}
\subsection{Naming convention}
\label{sec:orgec1d30d}
\texttt{vp\_bI\_nEE\_rDDDD.dat} where
\begin{itemize}
\item \texttt{bI} describes how \(\beta\) value was obtained:
I = 0 : \(\beta = 0\)
I = 1 : \(\beta = (1 - \sqrt{1-\rho^2})/\rho\)
I = 2 : \(\beta = (1 - 2\sqrt{1-\rho^2})/\rho\)
\item \texttt{nEE} indicates that \(n = 2^{\rm EE}\).
\item \texttt{rDDDD} represents the value of \(\rho = 0.{\rm DDDD}\).
\end{itemize}
\subsection{Example}
\label{sec:org83e5060}
An example data file \texttt{vp\_b0\_n07\_r5000.dat} may look like
\begin{center}
\begin{tabular}{rrl}
line & file & note\\
\hline
1 & 128 & n\\
2 & 0.5000 & rho\\
3 & 0.0000 & beta\\
4 & 2.7814117251577763\,(-01) & U\\
5 & 2.3675575948962696\,(-01) & a\(_{\text{1}}\)\\
6 & -6.6992137885540828\,(-02) & a\(_{\text{2}}\)\\
\vdots & \vdots & \vdots\\
131 & 2.8142318944085296\,(-20) & a\(_{\text{n-1}}\)\\
\end{tabular}
\end{center}

Our \texttt{data} directory looks like this:
\begin{minted}[]{shell}
tree ../data
\end{minted}

Here is one of the actual data file:
\begin{minted}[]{shell}
cat ../data/vp_b0_n07_r1000.dat
\end{minted}

\section{LINPACK}
\label{sec:org04b268f}
\subsection{Useful subroutines: double precision}
\label{sec:org8da3aed}
\begin{itemize}
\item DGECO: calculates condition number
\item DGEDI: calculates determinant
\item DGESL: solves A*X = B
\end{itemize}
\subsection{Example}
\label{sec:org6eb6ca4}
\begin{minted}[]{fortran}
integer :: n                  ! order of matrix A = JAC
integer :: lda = n            ! leading dimension of A
integer :: ipvt(n)            ! pivot indices
integer :: job = 0
real(rk) :: rcond
real(rk) :: z(n)
real(rk) :: jac(n,n)
!
! linear algebra routines (LINPACK) ----------------------------
!
call DGECO(jac, lda, n, ipvt, rcond, z)
! LU-factors JAC and estimates RCOND;
! JAC, on return, provides L and U
! IPVT is the pivot indices;
! Z is a work vector;
call DGESL(jac, n, n, ipvt, res, job)
! Solves JAC*X = RES; on return, RES is the solution;
! JOB = 0 for non-transposed problem
\end{minted}
\subsection{Notes}
\label{sec:orgb0a6848}
\begin{itemize}
\item When using Burkardt's \texttt{linpack\_d.f90}, make sure to link \texttt{lapack}
as it is not self-contained.
\begin{verbatim}
gfortran -o main.exe main.f90 linpack_d.f90 -framework Accelerate
\end{verbatim}
\item However, the quadruple precision library \texttt{linpack\_q.f90} is
self-contained:
\begin{verbatim}
gfortran -o main.exe main.f90 linpack_q.f90
\end{verbatim}
\item When using \texttt{linpack}, compile with either
\begin{itemize}
\item \href{file:///Users/tae/Dropbox/src/linpack.f}{linpack.f} (Fortran77)
\item \href{file:///Users/tae/Dropbox/src/linpack\_d.f90}{linpack\(_{\text{d.f90}}\)} (Fortran90) with \texttt{-framework Accelerate} flag.
\end{itemize}
Even if the main program follows Fortran90 standards, \texttt{linpack.f}
works seamlessly.
\item \textbf{Update} The source files \texttt{linpack*} are simply collection of
routines (dependencies) required for \texttt{DGECO}, \texttt{DGEDI}, and
\texttt{DGESL}. Some of them are again dependent on some routines of
\texttt{blas} library. The required routines are identified and combined
into a single source file for both \texttt{linpack} and \texttt{lapack}.
\end{itemize}

\section{LAPACK (modern)}
\label{sec:orgebc566e}
\subsection{Using LAPACK in Mac}
\label{sec:orgeed60b3}
Mac supplies a copy of LAPACK compiled and optimized for Apple hardwares
and it is easily available as a library. In order to link/load the
library, include \texttt{-framework Accelerate} compilation flag, e.g.,
\begin{minted}[]{shell}
gfortran -o myprog.exe myprog.f90 -framework Accelerate
\end{minted}

The library is located in the system directory \href{file:///System/Library/Frameworks/Accelerate.framework/}{/System/Library/Framework}.

\subsection{Useful subroutines and their usage}
\label{sec:orgf3d08d5}
Let \(A \in \mathbb{R}^{n \times n}\).
\begin{itemize}
\item DGETRF: LU-factorization of \(A\)
\item DGECON: calculates condition number of \(A\)
\item DGETRI: calculates the inverse \(A^{-1}\) using LU-decomposition
\item DGETRS: solves \(A x = b\) via Gaussian elimination, i.e. LU-factorization
\end{itemize}

\begin{minted}[]{fortran}
integer :: m                  ! number of rows
integer :: n                  ! number of columns
real(8) :: a(lda,n)           ! matrix A; on exit, factors L and U
integer :: lda                ! leading dimension of A
integer :: ipiv(n)            ! ivot indices, dimension = min(m,n)
integer :: info               ! 0 for successful exit
call dgetrf(m, n, a, lda, ipiv, info)
\end{minted}

\begin{minted}[]{fortran}
character*1 :: norm            ! '1' for 1-norm; 'I' for infinity-norm
integer :: n                   ! order of matrix
real(8) :: a(lda,n)            ! matrix/array A
integer :: lda                 ! leading dimension of A
real(8) :: anorm               ! 1-norm or infinity-norm of A; it is an input
real(8) :: rcond               ! rcond  = 1/(norm(A)norm(A^{-1}))
real(8) :: work(4*n)           ! double array
integer :: iwork(n)            ! integer array
integer :: info                ! 0 for successful exit
call dgecon(norm, n, a, lda, anorm, rcond, work, iwork, info)
\end{minted}

\begin{minted}[]{fortran}
character*1 :: trans        ! form of the S.O.E.; 'N' for no transpose
integer :: n                ! order of matrix A
integer :: nrhs             ! number of right-hand side
real(8) :: a(lda,n)         ! matrix A
integer :: lda              ! leading dimension of A
integer :: ipiv(n)          ! pivot indices
real(8) :: b(ldb,nrhs)      ! right-hand side; on exit, returns the solution X
integer :: ldb              ! leading dimension of A
integer :: info             ! 0 for successful exit
call dgetrs(trans, n, nrhs, a, lda, ipiv, b, ldb, info)
\end{minted}
\begin{itemize}
\item \textbf{Note.} One of the inputs, \texttt{anorm}, for \texttt{dgecon} must be calculated
before calling the routine. A function calculating matrix
infinity-norm has been included in my module, currently named as
\texttt{mnorm}.
\end{itemize}

\subsection{Example snippet}
\label{sec:org0669f4f}

\begin{minted}[]{fortran}
! for lapack routines
integer :: n
integer :: lda = n
integer :: ldb = n
integer :: lwork = 4*n
integer :: info
integer :: ipiv(n)
integer :: iwork(n)
real(rk) :: anorm
real(rk) :: rcond
real(rk) :: work(lwork)

! linear algebra routines (LAPACK) -------------------------------
anorm = mnorm(n, n, jac)   ! matrix infinity-norm
! calculating infinity norm of matrix JAC
call DGETRF(n, n, jac, lda, ipiv, info)
call DGECON('O', n, jac, lda, anorm, rcond, work, iwork, info)
call DGETRS('N', n, 1, jac, lda, ipiv, res, ldb, info)
\end{minted}

\section{FFTW}
\label{sec:orgbf523ec}
\subsection{Installation and basic usage}
\label{sec:org4ab39c8}
On Mac, I used \texttt{homebrew}
\begin{minted}[]{shell}
brew install fftw
\end{minted}

The files \texttt{libfftw3xxx.a} are saved in \href{file:///usr/local/lib/}{/usr/local/lib} directory. At link time, use \texttt{-l} flag as follows:
\begin{minted}[]{shell}
gfortran -o myprog.exe myprog.f90 -lfftw3
\end{minted}

The program file \texttt{myprog.f90} should contain a line
\begin{minted}[]{fortran}
include "fftw3.f90"
\end{minted}
which declares variables used.

On my linux machine running ArchLinux, I installed it using \texttt{packer}:
\begin{minted}[]{shell}
packer fftw fftw-quad
\end{minted}

\subsection{Variable declaration}
\label{sec:org77138c2}
The file \texttt{fftw3.f90} declares variables needed for execution of \texttt{fftw} routines:
\begin{minted}[]{shell}
cat ../src/fftw3.f90
\end{minted}

\subsection{Forward and backward 1-D (complex) DFT routines}
\label{sec:org82f6eeb}
The \textbf{forward DFT} of 1-D complex array \(X\) of size \(n\) calculates an
array \(Y\) of the same dimension where
\[
Y_k = \sum_{j=0}^{n-1} X_j e^{-2\pi i j k / n} \,.
\]

The \textbf{backward DFT} computes
\[
Y_k = \sum_{j=0}^{n-1} X_j e^{2\pi i j k / n} \,.
\]

Note that \texttt{fftw} computes unnormalized transforms. So Fourier series
coefficients can be approximated using the forward DFT with \(1/n\). The
inverse discrete Fourier transform is numerically calculated with the
backward DFT without any normalization.

Note also that an output of the forward DFT are ordered so that the
first half of the output corresponds to the positive modes while the
second half to the negative ones in backwards order; this is due to
the \$n\$-periodicity of \(Y_k\) in its index.

\textbf{Example.} When \(n = 8\), we have the following correspondence between indices (\(k\)), mode numbers, and Fortran indices:

\begin{center}
\begin{tabular}{rrr}
k & mode & fortran\\
\hline
0 & 0 & 1\\
1 & 1 & 2\\
2 & 2 & 3\\
3 & 3 & 4\\
4 & Nyquist & 5\\
5 & -3 & 6\\
6 & -2 & 7\\
7 & -1 & 8\\
\end{tabular}
\end{center}

In general:
\begin{center}
\begin{tabular}{lll}
k & mode & fortran\\
\hline
0 & 0 & 1\\
1 & 1 & 2\\
: & : & :\\
n/2-1 & n/2-1 & n/2\\
n/2 & Nyquist & n/2+1\\
n/2+1 & -n/2+1 & n/2+2\\
: & : & :\\
n-1 & -1 & n\\
\end{tabular}
\end{center}

\textbf{Snippet.}
\begin{minted}[]{fortran}
  implicit none
  include "fftw3.f90"
  integer ( kind = 4 ), parameter :: n = 100

  complex ( kind = 8 ) in(n)
  complex ( kind = 8 ) in2(n)
  complex ( kind = 8 ) out(n)
  integer ( kind = 8 ) plan_backward
  integer ( kind = 8 ) plan_forward
!
!  Make a plan for the FFT, and forward transform the data.
!
  call dfftw_plan_dft_1d_ ( plan_forward, n, in, out, FFTW_FORWARD, FFTW_ESTIMATE )
  call dfftw_execute_ ( plan_forward )
  out = out/real(n, kind=8)     ! normalization
!
!  Make a plan for the backward FFT, and recover the original data.
!
  call dfftw_plan_dft_1d_ ( plan_backward, n, out, in2, FFTW_BACKWARD, FFTW_ESTIMATE )
  call dfftw_execute_ ( plan_backward )
  print *, maxval(abs(in-in2))  ! compare the recovered data against the original
!
!  Discard the information associated with the plans.
!
  call dfftw_destroy_plan_ ( plan_forward )
  call dfftw_destroy_plan_ ( plan_backward )
\end{minted}

\subsection{Forward and backward 2-D (complex) DFT routines}
\label{sec:orgc22eeb7}
The \textbf{forward DFT} of \(n \times n\) 2-D complex array \(X\) calculates an
array \(Y \in \mathbb{C}^{n \times n}\) where
\[
Y_{j,k} = \sum_{m=0}^{n-1} \sum_{l=0}^{n-1} X_{l,m} e^{-2\pi i (jl+km)/n} \,.
\]

The \textbf{backward DFT} computes
\[
Y_{j,k} = \sum_{m=0}^{n-1} \sum_{l=0}^{n-1} X_{l,m} e^{2\pi i (jl+km)/n} \,.
\]

Note that these are simply the separable product of 1-D transforms
along each dimension of the array \(X\), that is, along the columns and
rows of \(X\).

Once again, \texttt{fftw} computes unnormalized transforms and so double
Fourier series coefficients can be approximated using the forward DFT
with \(1/n^2\). The inverse discrete Fourier transform is numerically
calculated with the backward DFT without any normalization.

\textbf{Snippet}
\begin{minted}[]{fortran}
implicit none
include "fftw3.f90"
integer ( kind = 4 ), parameter :: n = 100
integer ( kind = 4 ) :: i, j
complex ( kind = 8 ) :: in(n)
complex ( kind = 8 ) :: in2(n,n)
complex ( kind = 8 ) :: out(n)
complex ( kind = 8 ) :: out2(n,n)
complex ( kind = 8 ) :: X(n,n)
complex ( kind = 8 ) :: Y(n,n)
integer ( kind = 8 ) :: plan_backward
integer ( kind = 8 ) :: plan_forward

! Method 1: double 1d-fft's
do j = 1,n                    ! along columns
   in = X(:, j)
   call dfftw_plan_dft_1d_ &
        ( plan_forward, n, in, out, FFTW_FORWARD, FFTW_ESTIMATE)
   call dfftw_execute_ ( plan_forward )
    X(:, j) = out/real(n, rk)
end do
do i = 1,n                    ! then along rows
   in = X(i, :)
   call dfftw_plan_dft_1d_ &
        ( plan_forward, n, in, out, FFTW_FORWARD, FFTW_ESTIMATE)
   call dfftw_execute_ ( plan_forward )
   Y(i, :) = out/real(n, rk)
end do
call dfftw_destroy_plan_ ( plan_forward )

! Method 2: 2d-fft
in2 = X
call dfftw_plan_dft_2d_ ( plan_forward, n, n, in2, out2, FFTW_FORWARD, &
     FFTW_ESTIMATE)
call dfftw_execute_ ( plan_forward )
Y = out2/real(n**2, rk)
call dfftw_destroy_plan_ ( plan_forward )
\end{minted}

\subsection{Shifted FFT}
\label{sec:org9971c98}
Using the approximating nature of DFT on physical data against the coefficients of Fourier expansion, we may utilize FFT routines on data obtained on half-step shifted grids on \([0, 2\pi)\).

For the sake of illustration, consider the 1-D FFT situation where \(X^{(s)} \in \mathbb{C}^{N}\) is a vector of point values at \(2\pi(j+1/2)/N\) for \(0 \le j < N\) and \(Y^{(s)}\) is the result of the forward FFT on \(X^{(s)}\), i.e.,
\[
Y^{(s)}_j = \frac{1}{N} \sum_{k=0}^{N-1} X^{(s)}_k e^{-2 \pi i j k /N} \,.
\]

By adjusting the phase of complex exponentials, we can interpret them in term of approximate Fourier coefficients of the underlying function for the \(X^{(s)}\) data. Keeping in mind the aliasing errors associated with the discrete Fourier transforms, i.e., the \$N\$-periodicity over the index \(j\), we observe that

\begin{itemize}
\item For \(0 \le j < N/2\),
\[
   Y^{(s)}_j \approx Y_j e^{- i j/N} \,,
   \]
\item For \(N/2 < j < N\),
\[
   Y^{(s)}_j \approx Y_j e^{- i (j-N)/N} \,,
   \]
\end{itemize}

The second case was considered with
\section{Compiling with \texttt{gfortran}}
\label{sec:orgb0feddd}
\section{Makefile}
\label{sec:org3302c69}
\subsection{Editing a make file in Emacs/Org-mode}
\label{sec:org1f125b7}
\begin{itemize}
\item When a \texttt{make} source code written in org-mode src block is tangled,
tabs are converted to spaces. One can manually \texttt{M-x tabify} the
entire file.
\item A makefile whose file name is not \texttt{Makefile} will not be in \texttt{makefile-mode} automatically. Set the mode by \texttt{M-x makefile-mode}.
\item In order to run a makefile with filename other than \texttt{Makefile}, use \texttt{make -f filename}.
\end{itemize}

\subsection{Some idea from StackOverflow}
\label{sec:orgff454f0}
From \href{https://stackoverflow.com/questions/8855896/specify-directory-where-gfortran-should-look-for-modules}{Specify directory where gfortran should look for modules}

\begin{verbatim}
You can tell gfortran where your module files (.mod files) are located with the -I compiler flag. In addition, you can tell the compiler where to put compiled modules with the -J compiler flag. See the section "Options for directory search" in the gfortran man page.

I use these to place both my object (.o files) and my module files in the same directory, but in a different directory to all my source files, so I don't clutter up my source directory. For example,

SRC = /path/to/project/src
OBJ = /path/to/project/obj
BIN = /path/to/project/bin

gfortran -J$(OBJ) -c $(SRC)/bar.f90 -o $(OBJ)/bar.o
gfortran -I$(OBJ) -c $(SRC)/foo.f90 -o $(OBJ)/foo.o
gfortran -o $(BIN)/foo.exe $(OBJ)/foo.o $(OBJ)/bar.o
While the above looks like a lot of effort to type out on the command line, I generally use this idea in my makefiles.

Just for reference, the equivalent Intel fortran compiler flags are -I and -module. Essentially ifort replaces the -J option with -module. Note that there is a space after module, but not after J.
\end{verbatim}

\section{Fortran 90 programming tips}
\label{sec:org8a43dbf}
\subsection{Notes on precision in fortran90+}
\label{sec:orgbb99309}
At the beginning of program, set real precision, which is of integer type, to be
\begin{itemize}
\item 4 : single
\item 8 : double
\item 16 : quadruple
\end{itemize}
For example, declare
\begin{minted}[]{fortran}
integer, parameter :: rp = 16
\end{minted}

Then the precision of a real or complex variable can be declared by:
\begin{minted}[]{fortran}
real(kind=rp) :: x
complex(kind=rp) :: z
\end{minted}
or simply by
\begin{minted}[]{fortran}
real(rp) :: x
complex(rp) :: z
\end{minted}

In the body of program, the precision of a floating point number can be set by suffixing with the precision parameter, e.g.
\begin{itemize}
\item 1.0\(_{\text{4}}\) : single, same as 1.0
\item 1.0\(_{\text{8}}\) : double, same as 1.d0
\item 1.0\(_{\text{16}}\) : quadruple
\end{itemize}
Once the precision is stored in the variable, say  \texttt{rp}, one can simply write \texttt{1.0\_rp}.

The precision of outputs of an intrinsic function is determined by that of its input(s). This way, we can avoid using old \texttt{d}-variations, e.g., \texttt{dcos}, \texttt{dsin}, \texttt{dabs}, etc.

A complex number of certain precision can be constructed using \texttt{cmplx} function with the following syntax:
\begin{minted}[]{fortran}
z = cmplx( x, y, rp )
\end{minted}

\subsection{Nice tips on using \texttt{read} and \texttt{write} functions}
\label{sec:org014bb6d}

The following is an example of using \texttt{write} function to assimilate the functionality of \texttt{MatLab}'s \texttt{sprintf}.

\begin{minted}[]{fortran}
program play
  implicit none
  integer :: m
  real(8) :: rho

  interface
     function genfilename(n, rho, betaopt)
       integer, intent(in) :: n
       real(8), intent(in) :: rho
       integer, intent(in) :: betaopt
     end function genfilename
  end interface

end program play

function genfilename(n, rho, betaopt)
  implicit none
  ! declaring arguments
  integer, intent(in) :: n
  real(8), intent(in) :: rho
  integer, intent(in) :: betaopt
  ! output
  character(len=28) :: genfilename
  ! local variables
  character(len=*), parameter :: fmt = trim('(a, i1, a, i0.2, a, i0.4, a)')
  integer :: log2n
  log2n = int(log(real(n))/log(real(2)))
  write( genfilename, fmt ) &
       '../data/vp_b', betaopt, '_n', log2n, '_r', int(1000*rho), '.dat'
end function genfilename
\end{minted}

\subsection{\texttt{linspaceh} function:}
\label{sec:orgf4426f7}
The function \texttt{linspaceh(a, b, h)} constructs a vector of uniformly spaced-out points between \texttt{a} and \texttt{b} with gap \texttt{h}. In case \texttt{b-a} is not a (numerical) multiple of \texttt{h}, then the gap between \texttt{b} and the one before will be smaller than \texttt{h}. This function is included in \href{file:///Users/tae/Google\%20Drive/VP\_fortran/src/mymod.f90}{mymod.f90}.

\begin{minted}[]{fortran}
program main
  implicit none
  real(8) :: rho0
  real(8) :: rho1
  real(8) :: drho
  real(8), dimension(:), allocatable :: rhodpt
  integer :: i

  interface
     function linspaceh(a, b, h)
       real(8), intent(in) :: a, b, h
       real(8), dimension(:), allocatable :: linspaceh
     end function linspaceh
  end interface
  rho0 = 0.1D0
  rho1 = 0.53D0
  drho = 0.05D0
  rhodpt = linspaceh(rho0, rho1, drho)
  do i = 1, size(rhodpt)
     write(*, '(i3, 2x, f6.4)') i, rhodpt(i)
  end do
  print *, rhodpt, "hello world"
  deallocate (rhodpt)
end program main

function linspaceh(a, b, h)
  implicit none
  real(8), intent(in) :: a, b, h
  real(8), parameter :: eps = epsilon(1.d0)
  integer :: i, n
  real(8), dimension(:), allocatable :: linspaceh
  n = int((b-a)/h)
  ! if (abs((b-a)/h-n)<eps) then  -----------> it doesn't work
  ! if ( abs( (b-a)-n*h ) < eps ) then ------> this works
  if ( abs(mod(b-a, h)) < eps ) then ! note we compare against the
     ! machine epsilo n
     allocate(linspaceh(n+1))
     linspaceh = (/ (a + h*i, i=0,n) /)
  else
     allocate(linspaceh(n+2))
     linspaceh = (/ (a+h*i, i=0,n), b /) ! note how a vector and a
                                         ! number are concatenated
  end if
end function linspaceh
\end{minted}

\subsection{I/O formatting}
\label{sec:org9cdbc28}
\subsubsection{\texttt{print}}
\label{sec:orgbc969e9}
Printing out to terminal: \texttt{print format\_specifier, i/o\_list},
e.g. \texttt{print *, 'hello world'}
\subsubsection{\texttt{write}}
\label{sec:org98ef15c}
\begin{minted}[]{fortran}
implicit none
real(8) :: pi = acos(-1.0d0)
integer :: n = 256
write(*, '( i30   )') n          ! integer
write(*, *) pi                   ! free format
write(*, '( f30.16)') pi         ! decimal
write(*, '( g30.16)') pi         ! whichever is nice
write(*, '( e30.16)') pi         ! exponential notation
write(*, '(es30.16)') pi         ! scientific notation
write(*, '(en30.16)') pi         ! engineering notation
write(*, '( e30.16E3)') pi       ! exponential notation, more spaces for exponents
write(*, '(2e30.16)') pi, 2*pi   ! concatenation of two
\end{minted}

\subsection{Arrays - basics}
\label{sec:org36ec673}
\begin{itemize}
\item In Fortran 90+, one can construct arrays with inline do-loops, a.k.a., implied do-loops. For example,
\begin{minted}[]{fortran}
v = (/ (i, i=1,10) /)
w = [ (j, j=1,100,2) ]
\end{minted}

\item Of course, \texttt{i} and \texttt{j} need to be declared integers and \texttt{v} and \texttt{w}
as integer/real/complex arrays of appropriate dimensions. Note
below how types are cast:
\begin{minted}[]{fortran}
program arrays
  implicit none
  integer, parameter :: m=2, n=5
  integer :: i
  integer :: v_int(n), inner, outer(n,n)
  real :: v_real(n), a(m,n), a1(m,n), b(n,m), b1(n,m), c(m,m), d(n,n)
  complex :: v_cmplx(n)

  ! constructing vectors using inline do-loops
  v_int = [ (i, i = 1,n) ]
  v_real = [ (i, i = 1,n) ]
  v_cmplx = [ (cmplx(i, sqrt(real(i))), i = 1,n) ]

  ! type-casting
  do i = 1,n
     print *, v_int(i), v_real(i), v_cmplx(i)
  end do

  ! 2-D array using do-loop
  do i = 1,n
     a(1:m,i) = (/ i, i+1 /)
  end do

  ! 2-D array using reshape and implied do-loop
  a1 = reshape( [(i, i=1,m*n)], [m,n] )

  ! transpose
  b = transpose(a)

  ! reshaping
  b1 = reshape( a1, [n,m] )

  ! scalar multiplication, inner product, outer product
  v_int = 2*v_int               ! scalar multiplication
  inner = dot_product(v_int, v_int) ! inner product
  outer = spread(v_int, 2, n)*spread(v_int, 1, n)
  ! spread behaves like MatLab's repmat function

  print *, ''
  do i = 1,n
     print *, v_int(i)
  end do
end program arrays
\end{minted}

\item \texttt{size} and \texttt{shape}
\end{itemize}
\subsection{Arrays - assignment}
\label{sec:org7b8765f}
Many functions on arrays behave similarly to those of \texttt{MatLa}.

\begin{minted}[]{fortran}
implicit none
integer, parameter :: n = 5
integer, parameter :: rk = 8
integer :: i
real(rk) :: v(n), w(n), A(n,n)

A = 0.0_rk                    ! creating zero matrix
write(*, '(a)') 'Creating zero matrix'
write(*, '(a/)') 'A = 0.0_rk  yields '
write(*, '(5f8.4)') A

v = (/ (i, i=1,n) /)
w = (/ (i, i=n,1,-1) /)
write(*, '(/a)') 'Defining vectors'
write(*, '(5f8.4)') v
write(*, '(5f8.4)') w

write(*, '(/a)') 'Assigning columns of A:'
do i = 1,n
   A(:,i) = v**i
end do
do i = 1,n
   write(*, '(5g12.4)') A(i, :)
end do
\end{minted}

\subsection{Arrays: \texttt{transpose} and \texttt{spread}}
\label{sec:org2ace33d}
\begin{minted}[]{fortran}
implicit none
integer, parameter :: n = 5
integer, parameter :: rk = 8
integer :: i
real(rk) :: v(n), w(n), A(n,n), B(n,n)

v = (/ (i, i=1,n) /)
w = (/ (i, i=n,1,-1) /)
write(*, '(/a)') 'Defining vectors'
write(*, '(5f8.4)') v
write(*, '(5f8.4)') w

write(*, '(/a)') 'spread: equivalent of repmat in matlab'
write(*, '(a)') '   syntax: spread( array, dim, ncopies )'
A = spread(v, 1, n)
B = spread(v, 2, n)

write(*, '(/a)') '  A = '
do i = 1,n
   write(*, '(5g12.4)') A(i, :)
end do

write(*, '(/a)') '  B = '
do i = 1,n
   write(*, '(5g12.4)') B(i, :)
end do

B = transpose(A)
write(*, '(/a)') 'Transpose of a matrix'
write(*, '(a)') '    syntax: transpose( array )'
write(*, '(/a)') '  A transpoe = '
do i = 1,n
   write(*, '(5g12.4)') B(i, :)
end do
\end{minted}

\subsection{Arrays: general concatenation using \texttt{reshape}}
\label{sec:org198c7d8}
\begin{minted}[]{fortran}
implicit none
integer, parameter :: n = 5
integer, parameter :: rk = 8
integer :: i
real(rk) :: v(n), w(n), A(n,n), B(n,n), C(2*n,n), D(n,2*n)

v = (/ (i, i=1,n) /)
w = (/ (i, i=n,1,-1) /)

A =  reshape((/ (v**i, i=1,n) /), [n,n])
write(*, '(/a)') 'Concatenation using reshape function'
do i = 1,n
   write(*, '(5g12.4)') A(i, :)
end do

B =  transpose(reshape((/ (v**i, i=1,n) /), [n,n]))
write(*, '(/a)') 'Concatenation using reshape function'
do i = 1,n
   write(*, '(5g12.4)') B(i, :)
end do

C(1:n, :) = A
C(n+1:2*n, :) = B(n:1:-1, :)
write(*, '(/a)') 'Concatenating two matrices'
do i = 1,2*n
   write(*, '(5g12.4)') C(i, :)
end do
\end{minted}

\subsection{Arrays: trick to calculate maximal value of an array using \texttt{reshape}}
\label{sec:orgc1f2cdc}
\begin{minted}[]{fortran}
implicit none
real(8) :: a(3,3)

a(1,1) = 1.0D+00
a(1,2) = 2.0D+00
a(1,3) = 3.0D+00

a(2,1) = 4.0D+00
a(2,2) = 5.0D+00
a(2,3) = 6.0D+00

a(3,1) = 7.0D+00
a(3,2) = 8.0D+00
a(3,3) = 0.0D+00

print *, maxval( reshape(a, [size(a)]) ) ! turn a into a vector
! the 2nd arg of reshape should be an array of rank 1
\end{minted}

\subsection{Characters}
\label{sec:org2db8d92}
\subsubsection{Example}
\label{sec:orgbbc7a5a}
\begin{minted}[]{fortran}
program character
  implicit none
  character(len=30) :: fname

  fname = 'mydata.dat'
  print *, fname
  stop
end program character
\end{minted}
\subsubsection{Notes}
\label{sec:orgbc35629}
\begin{itemize}
\item Use \texttt{character(len=*)} when the length of a character string is not known.
\end{itemize}
\subsection{Important notes on \texttt{interface}}
\label{sec:orgf599c6c}
When one intends to input an array of arbitrary size into a routine, the \emph{assumed shape} technique turns out to be quite advantageous. Consider the following sample program:

\begin{minted}[]{fortran}
program dummy_array
  implicit none
  integer, dimension(10) :: a
  interface ! This interface block is necessary
     subroutine fill_array(a)
       integer, dimension(:), intent(out) :: a
     end subroutine fill_array
  end interface
  call fill_array(a)
  write(*, '(i4)') a
end program dummy_array

subroutine fill_array(a)
  implicit none
  ! argument declaration
  integer, dimension(:), intent(out) :: a
  ! local variable
  integer :: i, size_a
  size_a = size(a)
  do i = 1,size_a
     a(i) = i
  end do
end subroutine fill_array
\end{minted}

Note that the \texttt{interface} block is required. In the block, only the arguments to the routines need be type-cast.

In order to avoid writing such blocks over and over, utilize a module structure as follows:
\begin{minted}[]{fortran}
module mod
contains
  subroutine fill_array(a)
    implicit none
    ! argument declaration
    integer, dimension(:), intent(out) :: a
    ! local variable
    integer :: i, size_a
    size_a = size(a)
    do i = 1,size_a
       a(i) = i
    end do
  end subroutine fill_array
end module mod

program dummy_array
  use mod
  implicit none
  integer, dimension(10) :: a
  call fill_array(a)
  write(*, '(i4)') a
end program dummy_array
\end{minted}

\section{Testing area}
\label{sec:org311825d}
\begin{minted}[]{fortran}
implicit none
! argument declaration
integer, parameter :: n = 512
integer, parameter :: rk = 8
integer :: i, j, k
real(8) :: rho = 0.5d0
real(8) :: beta = 0.d0
real(8) :: pi = 4.d0*atan(1.d0)
real(8) :: M(n,n)
real(8) :: nu(n)
complex(8) :: eta(n), zeta(n), etah(n), integ(n,n), etahm(n,n)

nu = (/ (2*pi*i/n, i=0,n-1) /)
eta = exp( cmplx(0.0, 1.0, kind=8) * nu )
zeta = (eta-beta)/(1.0_rk-beta*eta)
etah = (rho**2/zeta+beta)/(1.0_rk+beta*(rho**2/zeta))
! write (*, '(2es25.10)') etah

integ = cmplx(spread(eta, 1, n) ** spread((/ (k-1, k=1,n) /), 2, n), kind=rk)
etahM = spread(etah, 1, n)
do j = 1,n
   integ = cmplx(integ*etahM, kind=rk)
   M(:,j) = 1/real(n, rk)*real(sum(integ, 2), rk)
end do
do i = 1,1
   do j = n-4,n
      ! write(*, '(2x, i4, 2x, i4, 2x, es16.5e3)') i, j, M(i,j)
      write(*, '(2x, i4, 2x, i4, 2x, e16.5E3)') i, j, M(i,j)
   end do
end do
\end{minted}

\begin{minted}[]{fortran}
implicit none
real(8) :: a(3,3), b(3)
integer :: i

a(1,1) = 1.0D+00
a(1,2) = 2.0D+00
a(1,3) = 3.0D+00

a(2,1) = 4.0D+00
a(2,2) = 5.0D+00
a(2,3) = 6.0D+00

a(3,1) = 7.0D+00
a(3,2) = 8.0D+00
a(3,3) = 0.0D+00

b = (/ 1.d0, 2.d0, 3.d0 /)
print *, size(a)
print *, reshape(a, (/9/))
\end{minted}
\end{document}
